\documentclass[12pt,openright,oneside,a4paper,ruledheader,pnumromarab,english]{abntex2}

\usepackage[alf,abnt-full-initials=yes,abnt-emphasize=bf]{abntex2cite} 	% Citações padrão ABNT
\usepackage{cmap}				
\usepackage{lmodern}					
\usepackage[T1]{fontenc}
\usepackage[utf8]{inputenc} %Codificacao do documento (conversão automática dos acentos)
\usepackage{lastpage}	        % Usado pela Ficha catalográfica
\usepackage{indentfirst}        % Indenta o primeiro parágrafo de cada seção.
\usepackage{color}				% Controle das cores
\usepackage[table]{xcolor}		% Pacote para a definição de novas cores
\usepackage{graphicx}			% Inclusão de gráficos
\usepackage{lipsum}
%\usepackage[brazilian,hyperpageref]{backref}	 % Paginas com as citações na bibl
%\usepackage[alf]{abntex2cite}	% Citações padrão ABNT
\usepackage{hyperref}
\usepackage{float} % PARA TENTAR AJUSTAR AS TABELAS E FIGURTAS
\usepackage{comment}	   		% Para criar blocos de Comentários
\usepackage{chngcntr}	   		% Para remover reset na contagem (chapter) de Figuras, Ilustrações, Quadros...
\usepackage{listings}			% Pacote com as configurações das sintaxe de Linguagens de programação
\usepackage{abntexufersa}		% Pacote com as configurações de estilo da UFERSA.
\usepackage{multirow}			% Pacote para tabelas com Celulas com Linhas Multiplas
\usepackage{booktabs}			% Pacote para tabelas organizadas
\usepackage{colortbl}			% Pacote para colorir linhas de tabelas
\usepackage{pgffor}
\usepackage{textcomp} %o simbolo de marca registrada



% ************************************************************
% ********** DEFINIÇÃO DE COMANDOS PERSONALIZADOS
% ************************************************************
% Definindo novas cores
\definecolor{verde}{rgb}{0.25,0.5,0.35}
\definecolor{jpurple}{rgb}{0.5,0,0.35}
\definecolor{darkgreen}{rgb}{0.0, 0.2, 0.13}
\definecolor{oldmauve}{rgb}{0.4, 0.19, 0.28}
\definecolor{lightgray}{gray}{0.9}

% ************************************************************
% ********** Configurando layout para mostrar codigos Java

\newcommand{\estiloJava}{
\lstset{
    language=Java,
    basicstyle=\ttfamily\small,
    keywordstyle=\color{jpurple}\bfseries,
    stringstyle=\color{red},
    commentstyle=\color{verde},
    morecomment=[s][\color{blue}]{/**}{*/},
    extendedchars=true,
    showspaces=false,
    showstringspaces=false,
    numbers=left,
    numberstyle=\tiny,
    breaklines=true,
    backgroundcolor=\color{cyan!10},
    breakautoindent=true,
    captionpos=b,
    xleftmargin=0pt,
    tabsize=2
}}


\titulo{Titulo do trabalho...}

\autor{Digite seu nome}
\local{Pau dos Ferros/RN}
\data{ANO}

\orientador{NomedoOrientador, Prof. Dr.}
\coorientador{nomew, Prof. Dr.}

\instituicao{%
 UNIVERSIDADE FEDERAL RURAL DO SEMI-ÁRIDO\par
 CENTRO MULTIDISCIPLINAR DE PAU DOS FERROS
 }
\curso{Bacharelado em Engenharia de Computação}
 
\tipotrabalho{Monografia (Trabalho de Conclusão de Curso)}

\preambulo{Monográfica apresentada a Universidade Federal Rural do Semi-árido - UFERSA, Campus Pau dos Ferros para a obtenção do título de Bacharel em Engenharia de Computação.}

\definecolor{blue}{RGB}{41,5,195}

\setlength{\parindent}{1.3cm}

\begin{document}

\frenchspacing 
\pretextual

\imprimircapa

\imprimirfolhaderosto*

% ************************************************************
% ********** FICHA CATALOGRÁFICA
% ************************************************************

\begin{fichacatalografica}
FICHA CATALOGRÁFICA AQUI... 

\textit{ELA SERÁ GERADA PELO SITE DA UFERSA}
	\vspace*{\fill}				% Posição vertical
%\begin{figure}
%    \centering
%    \includegraphics[scale=0.8]{imagens/Fcima.png}
    %\caption{Caption}/
%    \label{ficha}
%\end{figure}
\begin{center}					% Centralizado

% Aqui não se usou \vfill porque o \vfill é construído internamente com
% o comando \vspace. Espaços verticais no início da folha com \vspace
% são ignorados. Para que isto não ocorra deve-se usar o \vspace*
% \vspace*{\fill} é como se fosse um \vfill*
\vspace*{\fill}

%\textbf{Catalogação da Publicação na Fonte}\\[1ex]

%\textbf{Biblioteca Universitária Campus Pau dos Ferros (BIBLIOPF-UFERSA)}\\[0.5cm]

\vspace{2ex}

%\begin{figure}[H]
%    \centering
%    \includegraphics[scale=0.9]{ficha.png}
%    %\caption{Caption}
%    \label{ficha}
%\end{figure}

%\begin{tabular}{|p{0.9\linewidth}|} \hline
%\\
%Dias, Bruno José Nogueira.\\
%\hspace{1em}  \imprimirtitulo  / \imprimirautor. --
%	\imprimirlocal, \imprimirdata \\
%\hspace{1em} \pageref{LastPage} p. : il. \\
%\\
%\hspace{1em} \imprimirtipotrabalho . \\
%\\
%\hspace{1em} Orientador: \imprimirorientador \\
%\hspace{1em} Co-orientador: \imprimircoorientador \\
%\\
%\hspace{1em}	
%		1. Eletrocardiograma.
%		2. Transformada Wavelet.
%		3. Família Daubechies.
%		I. \imprimirorientador.
%		II. Universidade Federal Rural do Semi-árido. 
%		III. Campus Pau dos Ferros. Curso de Ciência e Tecnologia.
%		IV. Título\\ 	
%RN/UF/BIBLIOPF \hfill CDU 616.12-008.318 (Fornecido pela Biblioteca)\\ %\hline
%\end{tabular} 

\end{center}

\end{fichacatalografica}


% ************************************************************
% ********** FOLHA DE APROVAÇÃO
% ************************************************************

\begin{folhadeaprovacao}

   \begin{center}
    {\ABNTEXchapterfont\large\imprimirautor}

    \vspace*{\fill}\vspace*{\fill}
    {\ABNTEXchapterfont \imprimirtitulo}
    \vspace*{\fill}
    
    \hspace{.4\textwidth}
    \begin{minipage}{.55\textwidth}
        \imprimirpreambulo
    \end{minipage}%
    \vspace*{\fill}
   \end{center}
    
   Trabalho aprovado em \imprimirlocal, XX de \textit{Mês} de \textit{ano}:


%\begin{figure}[H]
%    \centering
%    \includegraphics[scale=0.4]{imagens/aafolha1.png}
%    %\caption{Caption}
%    \label{ass}
%\end{figure}
   \assinatura{\imprimirorientador \\ Presidente} 
   \assinatura{\imprimircoorientador \\ Professor Avaliador}
   \assinatura{Nome do terceiro avaliador \\ Professor Avaliador}
  
   \begin{center}
    \vspace*{0.5cm}
    {\imprimirlocal}
    \par
    {\imprimirdata}
    \vspace*{1cm}
  \end{center}
  
\end{folhadeaprovacao}

%==============================================================================
% Redefinição do conteúdo das listas de tabelas
%==============================================================================

\renewcommand{\listfigurename}{\normalsize\textbf{LISTA DE FIGURAS}}
\renewcommand{\listtablename}{\normalsize\textbf{ LISTA DE TABELAS}}

% ************************************************************
% ********** DEDICATÓRIA
% ************************************************************

\begin{dedicatoria}
   \vspace*{\fill}
	\begin{flushright}

\noindent \emph{A meu tio Rafael Reuber Nogueira.}

	\end{flushright}
\end{dedicatoria}
% ************************************************************
% ********** AGRADECIMENTOS
% ************************************************************
\begin{agradecimentos}

Inicialmente, agradeço aos Deuses 

\end{agradecimentos}

% ************************************************************
% ********** EPÍGRAFE
% ************************************************************

\begin{epigrafe}
    \vspace*{\fill}
	\begin{flushright}
	
%	\textit{O conhecimento nos faz necessário.} (Che Guevara)

\textit{O céu é o limite para as pessoas que buscam algo grande, para aquelas que querem algo maior, o infinito e além}

%	\textit{Elemento opcional colocado após os agradecimentos, \\ 
%	onde o autor apresenta uma citação (deve ser indicada a autoria)\\
%	relacionada com a matéria tratada no corpo do trabalho. \\ 
%	O autor pode também optar pela inserção de epígrafes \\ 
%	nas folhas de abertura das seções primárias.}\\
%	(John Bellamy Foster, The Rift)
	\end{flushright}

\end{epigrafe}

% ************************************************************
% ********** RESUMO
% ************************************************************

\begin{resumo} 
 
Aqui vou digitar o resumo do trabalho...

 \vspace{\onelineskip}
\noindent
 \textbf{Palavras-chaves}: Palavra1. Palavra2. Palavra3.
\end{resumo}

% ************************************************************
% ********** ABSTRACT
% ************************************************************

\begin{resumo}[Abstract]
\begin{otherlanguage*}{english}
Aqui vou digitar o Abstract do trabalho...

   \vspace{\onelineskip}
 
   \noindent 
   \textbf{Key-words}: Key1. Key2. Key3.
 \end{otherlanguage*}
\end{resumo}


\pdfbookmark[0]{\listfigurename}{lof}
\listoffigures*
\cleardoublepage

\pdfbookmark[0]{\listtablename}{lot}
\listoftables*
\cleardoublepage

\pdfbookmark[0]{\contentsname}{toc}
\tableofcontents
\cleardoublepage

%\mainmatter

\chapter{Introdução}

    \section{Motivação}
        Segundo \citeonline{IBGEeduca2010}, 

    \section{Objetivos}

        \subsection{Objetivo Geral}


        \subsection{Objetivos Específicos}

    \section{Organização}

        Este trabalho está organizado na seguinte divisão: O capítulo 1 apresenta a introdução, contextualização, motivação e os objetivos do trabalho.
    
        O capítulo 2 apresenta um levantamento de trabalhos relacionados,

        No capítulo 3 é apresentada a metodologia do desenvolvimento ...
    
        O capítulo 4 aborda as etapas de desenvolvimento do trabalho, como a construção do sistema de aquisição
    
        O capítulo 5 apresenta os testes iniciais e expõe os resultados obtidos durante cada uma das etapas do desenvolvimento e levanta a sua discussão;

        O capítulo 6 apresenta as considerações finais e os trabalhos futuros.

\chapter{ESTADO DA ARTE}

   
    \section{Trabalhos Relacionados}
%(BORTOLE, 2011)

        
%\chapter{ESTADO DA ARTE - PART2}

    \section{Outro Capitulo}
 
\chapter{Análise de Desempenho}

O presente capítulo apresenta a metodologia empregada no levantamento dos resultados do presente trabalho, bem como realiza a apresentação destes e sua análise.

\section{Materiais e Métodos}

\section{Resultados}



\section{Discussões}



\include{capitulos/cap5-Metologia}
\chapter{Conclusão}


\section{Trabalhos futuros}



\bibliographystyle{abntex2-alf}

\bibliography{refe}

%\postextual

\end{document}
